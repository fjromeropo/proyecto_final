\documentclass[a4paper,11pt]{article}
\usepackage[utf8]{inputenc}
\usepackage[spanish]{babel}
\usepackage{hyperref}
\usepackage{enumerate}
\usepackage{sectsty} %Este paquete se usa para poder cambiar el tamaño de las letras de los encabezados, y que no salgan demasiado grandes.
\begin{document}
\allsectionsfont{\normalsize}
\title{Cálculo de una instalación telesilla biplaza}
\author{Francisco Javier Romero Porras}
\maketitle
\begin{abstract}
Este es un trabajo que nos permite ver un ejemplo de como se debe proyectar un telesilla, de modalidad biplaza, de panza fija, para turistas no esquiadores. El trabajo se incluirá en el siguiente repositorio de git:\\
\url{https://github.com/fjromeropo/proyecto_final}
\end{abstract}
\textbf{Palabras claves:} proyecto, telesilla, biplaza, pilona, cable, , perfil.
\tableofcontents
\part{Introducción}
El objetivo del artículo es ejemplificar como debe proyectarse un telesilla, biplaza, de panza fija, para turistas no esquiadores, entre dos puntos distantes 550 m y con un desnivel total de 150 m. Se considerarán, además, las siguientes características para hacer el dimensionado de la instalación:\\
\begin{itemize}
\item La capacidad de la instalación será de $C=702 viajeros/hora$
\item La altura total de silla y pasajero es de 3.50 m
\item Se considera que no hay nieve, pero deben tenerse en cuenta los efectos dinámicos.
\item La masa de la silla es de 60 kg (biplaza), y se considera un peso de 80 kg por viajero.
\item Los volantes de ambas estaciones se encuentran a 4.00 m sobre el terreno.
\item El tipo de cable es de 6x36+1 Warrington - Seale
\item Por razones de explotación la propiedad impone las siguientes condiciones:
\begin{itemize}
\item La altura máxima de apoyos será de 12 m
\item El diámetro máximo del cable será de 32 mm
\item Se procurará que no se produzcan cambios de signo de la carga sobre los apoyos al pasar de la situación de cargado a descargado.
\item El motor y contrapeso deben disponerse en la estación inferior.
\end{itemize}
\item Los cálculos generales se realizarán en cuatro hipótesis de carga:
\begin{enumerate}[a)]
\item Ramal ascendente cargado y descendente cargado.
\item Ramal ascendente cargado y descendente descargado.
\item Ramal ascendente descargado y descendente cargado.
\item Ramal ascendente descargado y descendente descargado.
\end{enumerate}
\end{itemize}
Los cálculos detallados se realizarán solamente para la situación de ramal {\bf ascendente cargado,} en la hipótesis b), y ramal {\bf ascendente descargado,} en la hipótesis c).
Vamos a considerar los siguientes resultados:
\begin{enumerate}[1)]
\item Distancia entre vehículos.
\item Tensión máxima del cable.
\item Masa del contrapeso.
\item Altura del cable sobre el suelo en los apoyos (altura del apoyo).
\item Comprobación de adherencia.\\
Y para el {\bf ramal ascendente,} en ambas hipótesis:
\item Flecha de los vanos.
\item Comprobación de distancia al suelo.
\item Ángulos del cable en los extremos de los vanos y comprobación de la diferencia en situación de cargado a descargado.
\item Ángulo de deflexión en cada apoyo.
\item Resultante sobre los apoyos.
\item Ángulo de la resultante en cada apoyo.
\end{enumerate}
\part{Cálculos previos}
\part{Cálculos generales}
\section{Hipótesis 1}
\section{Hipótesis 2}
\section{Hipótesis 3}
\section{Hipótesis 4}
\part{Detalle cálculos de la instalación}
\section{Ramal ascendente cargado}
\section{Ramal descendente cargado}
\part{Perfil}
\end{document}